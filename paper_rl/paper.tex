%%
%% This is file `sample-sigconf.tex',
%% generated with the docstrip utility.
%%
%% The original source files were:
%%
%% samples.dtx  (with options: `sigconf')
%%
%% IMPORTANT NOTICE:
%%
%% For the copyright see the source file.
%%
%% Any modified versions of this file must be renamed
%% with new filenames distinct from sample-sigconf.tex.
%%
%% For distribution of the original source see the terms
%% for copying and modification in the file samples.dtx.
%%
%% This generated file may be distributed as long as the
%% original source files, as listed above, are part of the
%% same distribution. (The sources need not necessarily be
%% in the same archive or directory.)
%%
%% The first command in your LaTeX source must be the \documentclass command.
\documentclass[conference]{IEEEtran}
%% IEEE CNS addition:
\makeatletter
\def\ps@headings{%
\def\@oddhead{\mbox{}\scriptsize\rightmark \hfil \thepage}%
\def\@evenhead{\scriptsize\thepage \hfil \leftmark\mbox{}}%
\def\@oddfoot{}%
\def\@evenfoot{}}
\makeatother
\pagestyle{empty}

\usepackage[utf8]{inputenc}
\usepackage{multirow}
\usepackage[inline]{enumitem}
\usepackage{xcolor}
\usepackage{booktabs}
\usepackage{hyperref}
\usepackage{amsmath}
\usepackage{amssymb}
\usepackage{graphicx}
\usepackage[numbers]{natbib}
\usepackage{subfig}

\usepackage[nomain, toc, acronym]{glossaries}
\glsdisablehyper

\newcommand\note[2]{{\color{#1}#2}}
\newcommand\todo[1]{{\note{red}{TODO: #1}}}

%%
%% end of the preamble, start of the body of the document source.
\begin{document}

%%
%% The "title" command has an optional parameter,
%% allowing the author to define a "short title" to be used in page headers.
\title{SparseIDS: Learning Packet Sampling with Reinforcement Learning}

\author{\IEEEauthorblockN{Maximilian Bachl, Fares Meghdouri, Joachim Fabini, Tanja Zseby}
\IEEEauthorblockA{Technische Universität Wien\\
firstname.lastname@tuwien.ac.at}}

%\date{\today}

% \IEEEoverridecommandlockouts
% \IEEEpubid{\begin{minipage}[t]{\textwidth}\ \\[10pt]
%         \centering\normalsize{xxx-x-xxxx-xxxx-x/xx/\$31.00 \copyright 2018 IEEE}
% \end{minipage}}

% \renewcommand*{\bibfont}{\footnotesize}

\maketitle%

%\thispagestyle{plain}
%\pagestyle{plain}

\newacronym{ml}{ML}{Machine Learning}
\newacronym{dl}{DL}{Deep Learning}
\newacronym{aml}{AML}{Adversarial Machine Learning}
\newacronym{ids}{IDS}{Intrusion Detection System}
\newacronym{rnn}{RNN}{Recurrent Neural Network}
\newacronym{fgsm}{FGSM}{Fast Gradient Sign Method}
\newacronym{cw}{CW}{Carlini-Wagner method}
\newacronym{pgd}{PGD}{Projected Gradient Descent}
\newacronym{pdp}{PDP}{Partial Dependence Plot}
\newacronym{ars}{ARS}{Adversarial Robustness Score}
\newacronym{ttl}{TTL}{Time-to-Live}
\newacronym{dos}{DoS}{Denial-of-Service}
\newacronym{iat}{IAT}{Interarrival time}
\newacronym{rl}{RL}{Reinforcement Learning}

\newcommand{\ours}{SparseIDS}

\begin{abstract}

\glspl{rnn} have been shown to be valuable for constructing \glspl{ids} for network data. They allow analyzing network flows and already determining if a flow is malicious or not before it is over, making it possible to take action immediately. However, considering the large number of packets that have to be inspected, the question of computational as well as energy efficiency arises. We show that by using a novel \gls{rl}-based approach called \textit{\ours{}}, we can reduce the number of consumed packets by more than three fourths while keeping classification accuracy high. Comparing to various other sampling techniques, \ours{} consistently achieves higher classification accuracy with the same number of sampled packets. A major novelty of our \gls{rl}-based approach is that it can not only skip up to a predefined maximum number of packets like current approaches but can possibly skip arbitrarily many packets, saving even more computational resources for long sequences. Finally we build an automatic steering mechanism that can guide \ours{} to achieve a desired level of sparsity. 

\end{abstract}

%%
%% The abstract is a short summary of the work to be presented in the
%% article.
%\begin{abstract}
%  A clear and well-documented \LaTeX\ document is presented as an
%  article formatted for publication by ACM in a conference proceedings
%  or journal publication. Based on the ``acmart'' document class, this
%  article presents and explains many of the common variations, as well
%  as many of the formatting elements an author may use in the
%  preparation of the documentation of their work.
%\end{abstract}

\maketitle

\section{Introduction}
\label{sec:introduction}

There is a significant body of scientific work focusing on the detection of unwanted behavior in networks. In the past, a viable way of performing intrusion detection was to inspect the content of packets themselves and detect if a packet delivers potentially harmful content. More recently, with the increasing deployment of encryption, the focus now lies on features that are always available to network monitoring equipment like port numbers, protocol flags or packet sizes when encrypting above the transport layer.

Network communication is usually aggregated into \textit{flows}, which are commonly defined as a sequence of packets sharing certain properties. When analyzing flows, not only the aforementioned features are available but also features related to the timing of the individual packets. Various approaches have been proposed to extract features from flows and then perform anomaly detection with the extracted flows \cite{meghdouri_analysis_2018}.
While these approaches frequently work well, it is problematic that the whole flow has to be received first and only afterwards anomaly detection is applied, revealing attack flows. Thus, we design a network \gls{ids} that operates on a per-packet basis and decides if a packet is anomalous based on features that are available even for traffic that is encrypted above the transport layer, like for example TLS or QUIC.
At the same time, an \gls{rnn}-based \gls{ids} has the benefit of avoiding tedious feature engineering procedures, which derive statistical measures from the sequence of packet features, but providing any available information to the classifier.
\cite{hartl_explainability_2019} showed that such a system has similar performance to other flow-based anomaly detection systems but can detect anomalies \textit{before} the flow terminates and analyzed its security with respect to adversaries extensively. 

However, for practical use, not only classification accuracy and security are important but also computational efficiency. This follows from the fact that a rationally-acting company will only deploy an \gls{ids} if the cost that is saved by avoiding security breaches is higher than the cost of the \gls{ids} itself. The cost of the \gls{ids} can be split in the cost of purchase/installation, the cost of operating it and the cost of maintaining it. The goal we pursue here is to develop sampling strategies that do not significantly degrade classification performance while being able to only process a small fraction of the original data. Specifically, the sampling technique should
\begin{enumerate}
\item \textbf{choose optimally}, taking only the data that contain the most information and skip the less relevant ones. 
\item have a parameter that allows to \textbf{trade off} classification performance for sparsity.
\item be \textbf{independent of the classifier} so that any classifier can be used and that the classifier doesn't have to be aware of the sampling strategy. 
\item be \textbf{retrainable} so that the sampling can be continuously adapted depending on the current threat landscape. 
\item be able to \textbf{skip arbitrarily} many packets since network flows can be very long but only the first couple of packets are needed to decide whether an attack occurred or not and the remaining packets can be skipped altogether. 
\end{enumerate}

For this purpose, we develop the \gls{rl}-based \gls{ids} \textit{\ours{}} fulfills the above properties. We show that a significant number of packets can be skipped while accuracy doesn't drop significantly. Compared to other common sampling strategies, \ours{} performs better when being trained on the same number of packets. 

As \ours{} uses a parameter to trade off accuracy for sparsity, the resulting accuracy and sparsity are determined by this parameter. However, since it also might be desired to specify the sparsity itself, we propose a steering system which takes a desired sparsity as its input and modifies the parameter of the reinforcement learning in a control loop until the specified sparsity is reached. 

To encourage reproducibility and facilitate experimentation, we publicly release the source code, the data and the figures of this work at \url{https://github.com/CN-TU/adversarial-recurrent-ids/tree/rl}

\section{Related Work}

\cite{hartl_explainability_2019} build an LSTM-based recurrent classifier for Network Intrusion Detection and develop methods for explainability and to assess the security of recurrent classifiers. 

\cite{yu_learning_2017} propose using \gls{rl} to let an LSTM learn to skim text. Their technique does not aim to maximize sparsity but only lets the \gls{rl} optimize classification performance for the sequence as a whole. Thus, if it learns to skip elements it is only because it is better for achieving good classification performance, not because it wants to increase sparsity. 

\cite{campos_skip_2018} develop a technique that includes a \textit{skip gate} into LSTM or GRU \todo{citation needed} cells. It works without \gls{rl} but on the downside, their method can only be trained once and not be adapted afterwards and also it explicitly depends on the the implementation of the underlying classifier, while we want to be independent of it. 

\cite{seo_neural_2018} use an LSTM with one full state vector and one reduced one. At each step they decide whether to use the reduced or the full state vector and can reduce computational cost by using the reduced one. For network traffic we think that it is often not even necessary to have a reduced state vector since from some attack flows, already the first couple of packets might contain enough information and then the remaining packets of the flow do not have to be considered anymore at all. 

\cite{gui_long_2018} use \gls{rl} but their intention is not to skip samples but instead to make the \gls{rl} learn to choose a previous state that can aid at the current step. Thus it tries to help the \gls{rnn} with memorizing information from the past but does not actually take less data, which is the goal we pursue in this work. 

These proposed techniques focus their evaluation on text and hence do not fulfill certain criteria we set out in \autoref{sec:introduction}. Furthermore, all techniques that use \gls{rl} use discrete actions, meaning that there is a hyperparameter $k$, which influences the maximum jump that is possible. However, for a network flow, already after the second packet it might be obvious that a flow is an attack and thus the remaining packets can be completely ignored. It is thus beneficial to not have a fixed maximum step size but a continuous one. 

\section{Setup}

% \begin{table}[b]
% \caption{occurrence frequency of attack types.} \label{tab:occurrence}
% \centering
% \begin{tabular}{l r} \toprule
% Attack type & Proportion \\
% \midrule
% Normal                                                         & $0.747468$ \\
% DoS / DDoS:DoS Hulk                                            & $0.101014$ \\
% PortScan:PortScan - Firewall off                               & $0.069020$ \\
% DDoS:LOIT                                                      & $0.040784$ \\
% Infiltration:Dropbox download                                  & $0.032982$ \\
% DoS / DDoS:DoS GoldenEye                                       & $0.003224$ \\
% DoS / DDoS:DoS Slowhttptest                                    & $0.001819$ \\
% DoS / DDoS:DoS slowloris                                       & $0.001680$ \\
% Brute Force:SSH-Patator                                        & $0.001107$ \\
% Botnet:ARES                                                    & $0.000327$ \\
% Web Attack:XSS                                                 & $0.000293$ \\
% PortScan:PortScan - Firewall on                                & $0.000165$ \\
% Brute Force:FTP-Patator                                        & $0.000110$ \\
% Web Attack:Sql Injection                                       & $0.000006$ \\
% DoS / DDoS:Heartbleed                                          & $0.000001$ \\
% \bottomrule
% \end{tabular}
% \label{tab:occurrence}
% \end{table}

\begin{table}[b]
\caption{Flow occurrence frequency of attack types.}
\label{tab:occurrence}
\centering
%\subfloat[CIC-IDS-2017\label{fig:cicids17proportions}
%]{
\begin{tabular}{l r}
\toprule
Attack type & \hspace*{-4mm}Proportion \\ \midrule
DoS Hulk	&	10.10\%	\\
PortScan, Firewall	&	6.90\%	\\
DDoS LOIT	&	4.08\%	\\
Infiltration	&	3.30\%	\\
DoS GoldenEye	&	0.32\%	\\
DoS SlowHTTPTest	&	0.18\%	\\
DoS Slowloris	&	0.17\%	\\
Brute-force SSH	&	0.11\%	\\
Botnet ARES	&	0.03\%	\\
XSS attack	&	0.03\%	\\
PortScan, no Fw.	&	0.02\%	\\
Brute-force FTP	&	0.01\%	\\
SQL injection	&	$<$0.01\%	\\
Heartbleed	&	$<$0.01\%	\\
\bottomrule
\end{tabular}
%}{}
%\subfloat[UNSW-NB15\label{fig:unswnb15proportions}
%]{
%\begin{tabular}{l r}
%\toprule
%Attack type & \hspace*{-4mm}Proportion \\ \midrule
%Exploits	&	1.42\%	\\
%Fuzzers	&	1.01\%	\\
%Reconnaissance	&	0.58\%	\\
%Generic	&	0.21\%	\\
%DoS	&	0.19\%	\\
%Shellcode	&	0.08\%	\\
%Analysis	&	0.03\%	\\
%Backdoors	&	0.02\%	\\
%Worms	&	0.01\%	\\
%\bottomrule
%\end{tabular}
%}{}
\end{table}

We implemented a three-layer LSTM-based classifier with 128 neurons at each layer. As the input features we use
source port, destination port, protocol identifier, packet length, \gls{iat} to the previous packet in the flow, packet direction (i.e. forward or reverse path) and all TCP flags (0 if the flow is not TCP).
We omitted \gls{ttl} values, as they are likely to lead to unwanted prediction behaviour \cite{bachl_walling_2019}.  Among the used features, source port, destination port and protocol identifier are constant over the whole flow while the others vary.
During flow extraction we used the usual 5-tuple flow key, which distinguishes flows based on the protocol they use and their source and destination port and IP address.
We use Z-score normalization and a train/test split of 2:1.

%For evaluation, we use the \textit{CIC-IDS-2017} \cite{sharafaldin_toward_2018} and \textit{UNSW-NB15} \cite{moustafa_unsw-nb15:_2015} datasets which each include more than 2 million flows of network data, containing both benign traffic and a large number of different attacks. 
For evaluation, we use the \textit{CIC-IDS-2017} \cite{sharafaldin_toward_2018} dataset which includes more than 2 million flows of network data, containing both benign traffic and a large number of different attacks. 

Attacks contained in the datasets are shown in \autoref{tab:occurrence}.

%\begin{table}
%% Evaluated using
%% ./python learn.py --dataroot flows.pickle --net runs/Oct26_00-03-50_gpu/lstm_module_1284.pth --function test --batchSize 512
%% ./python learn.py --dataroot flows15.pickle --net runs/Oct28_15-41-46_gpu/lstm_module_997.pth --function test --batchSize 512
%% on Dec. 4th
%\caption{Performance metrics per packet and per flow.} \label{tab:performance_results}
%\centering
%\begin{tabular}{l r r r r} \toprule
%& \multicolumn{2}{c}{CIC-IDS-2017} & \multicolumn{2}{c}{UNSW-NB15} \\
%	&	Packet	&	Flow	&	Packet	&	Flow	\\	\midrule
%Accuracy	&	99.1\%	&	99.7\%	&	99.5\%	&	98.3\%	\\
%Precision	&	97.0\%	&	99.7\%	&	83.4\%	&	78.6\%	\\
%Recall	&	97.8\%	&	99.1\%	&	87.6\%	&	72.6\%	\\
%F1	&	97.4\%	&	99.4\%	&	85.5\%	&	75.5\%	\\
%Youden	&	97.2\%	&	99.0\%	&	87.3\%	&	71.9\%	\\
%\bottomrule
%\end{tabular}
%\end{table}

\section{Conclusion}

We have implemented a recurrent classifier based on LSTMs to detect network attacks, %While attack detection performance is not necessarily better than that of non-recurrent approaches,
which can already detect attacks before they are over. Furthermore, the recurrent approach allows us to inspect the influence of single packets on the detection performance and shows which packets are \textit{characteristic} for attacks.
Even though the interpretation of \glspl{rnn} poses several difficulties, we have demonstrated methods for gaining insights into the model's functioning.

We showed that even though our use case is very different from computer vision, adversarial samples can be found efficiently, even if only ostensibly unimportant features can be modified. We proposed the \gls{ars} for quantifying and comparing the adversarial threat for \glspl{ids}.
Deploying an adversarial training procedure, we could significantly reduce the adversarial threat.

\section*{Acknowledgements}
The Titan Xp used for this research was donated by the NVIDIA Corporation.

\renewcommand*{\bibfont}{\small}
\bibliographystyle{ieeetr}
\bibliography{bibliography}


\end{document}
\endinput
%%
%% End of file `sample-sigconf.tex'.
