%%
%% This is file `sample-sigconf.tex',
%% generated with the docstrip utility.
%%
%% The original source files were:
%%
%% samples.dtx  (with options: `sigconf')
%% 
%% IMPORTANT NOTICE:
%% 
%% For the copyright see the source file.
%% 
%% Any modified versions of this file must be renamed
%% with new filenames distinct from sample-sigconf.tex.
%% 
%% For distribution of the original source see the terms
%% for copying and modification in the file samples.dtx.
%% 
%% This generated file may be distributed as long as the
%% original source files, as listed above, are part of the
%% same distribution. (The sources need not necessarily be
%% in the same archive or directory.)
%%
%% The first command in your LaTeX source must be the \documentclass command.
\documentclass[sigconf,nonacm]{acmart}

\usepackage{multirow}

\settopmatter{printacmref=false} % Removes citation information below abstract
\renewcommand\footnotetextcopyrightpermission[1]{} % removes footnote with conference information in first column


%%
%% \BibTeX command to typeset BibTeX logo in the docs
\AtBeginDocument{%
  \providecommand\BibTeX{{%
    \normalfont B\kern-0.5em{\scshape i\kern-0.25em b}\kern-0.8em\TeX}}}


%%
%% Submission ID.
%% Use this when submitting an article to a sponsored event. You'll
%% receive a unique submission ID from the organizers
%% of the event, and this ID should be used as the parameter to this command.
%%\acmSubmissionID{123-A56-BU3}

%%
%% The majority of ACM publications use numbered citations and
%% references.  The command \citestyle{authoryear} switches to the
%% "author year" style.
%%
%% If you are preparing content for an event
%% sponsored by ACM SIGGRAPH, you must use the "author year" style of
%% citations and references.
%% Uncommenting
%% the next command will enable that style.
%%\citestyle{acmauthoryear}

%%
%% end of the preamble, start of the body of the document source.
\begin{document}

%%
%% The "title" command has an optional parameter,
%% allowing the author to define a "short title" to be used in page headers.
\title{Assignment 2}

%%
%% The "author" command and its associated commands are used to define
%% the authors and their affiliations.
%% Of note is the shared affiliation of the first two authors, and the
%% "authornote" and "authornotemark" commands
%% used to denote shared contribution to the research.
\author{Alexander Hartl (e01125115)}
\authornote{Both authors contributed equally to this research.}
%\email{alexander.hartl@tuwien.ac.at}

\author{Maximilian Bachl (e01100143)}
%\authornote[1]{Both authors contributed equally to this research.}
\authornotemark[1]
%\email{maximilian.bachl@gmail.com}

%%
%% By default, the full list of authors will be used in the page
%% headers. Often, this list is too long, and will overlap
%% other information printed in the page headers. This command allows
%% the author to define a more concise list
%% of authors' names for this purpose.
\renewcommand{\shortauthors}{Hartl and Bachl}

%%
%% The abstract is a short summary of the work to be presented in the
%% article.
%\begin{abstract}
%  A clear and well-documented \LaTeX\ document is presented as an
%  article formatted for publication by ACM in a conference proceedings
%  or journal publication. Based on the ``acmart'' document class, this
%  article presents and explains many of the common variations, as well
%  as many of the formatting elements an author may use in the
%  preparation of the documentation of their work.
%\end{abstract}

\maketitle

\section{Introduction}
The detection of attacks in data networks is a fundamental task in data security. Due to the considerable amount of data which has to be analyzed the use of machine learning techniques for this purpose seems natural and is increasingly deployed.

For research the invention and assessment of techniques for network anomaly detection poses many challenges. Particularly, a considerable number of features can be extracted from network data which might be beneficial for anomaly detection. For the training of models usually datasets are used which have been generated artificially in a controlled test environment. 

As a downside of this approach, it is unclear whether a machine learning model learns to classify based on characteristics that are inherent to the attacks which should be detected, or rather learns to classify based on patterns that were unintentionally created during dataset generation.

For a well-performing network anomaly detection technique it is therefore of utmost importance to study which patterns the technique looks at to distinguish attack traffic from normal traffic, and question if these explanations match with expert knowledge.

% TODO: think Fares didnt use this
In this document, we redo parts of a recent paper~\cite{fares} which bases on the CIC-IDS-2017 dataset \cite{Sharafaldin2018} for evaluating the performance of several feature vectors and machine learning techniques for accurate anomaly detection. We use explainability methods for investigating if the decisions the anomaly detectors untertake are reasonable.

Furthermore, we add a backdoor to the trained model and show that attack detection can efficiently be bypassed if the attacker had the ability to modify training data. Finally, we apply the same explainability methods to the backdoored model and show how such attack attempts might be recognized before any harm is done.

\section{Anomaly Detection Approaches} \label{sec:ml_approaches}
We used the CIC-IDS-2017~\cite{Sharafaldin2018} for our evaluations. As a first preprocessing step we extracted flow features from the dataset. We used features specificed by the CAIA feature vector~\cite{Williams2006} and slightly enhanced the included features. Table~\ref{tab:features} lists the features that are included for each flow.

We used Z-score normalization to process the data and 3-fold cross validation for evaluating the models' performances.

\begin{table*}
\renewcommand*{\arraystretch}{1.2} % TODO: better order
\caption{Flow features used for attack detection.}
\label{tab:features}
\begin{tabular}{l l l l l} \toprule
dstBytes	&	max\_srcPktLength	&	min\_dstPktIAT	&	srcPkts	&	\#dstTCPflag:ack	\\
dstPkts	&	max\_srcTTL	&	min\_dstPktLength	&	srcPort	&	\#dstTCPflag:cwr	\\
dstPort	&	mean\_dstPktIAT	&	min\_dstTTL	&	stdev\_dstPktIAT	&	\#dstTCPflag:fin	\\
duration	&	mean\_dstPktLength	&	min\_srcPktIAT	&	stdev\_dstPktLength	&	\#dstTCPflag:syn	\\
max\_dstPktIAT	&	mean\_dstTTL	&	min\_srcPktLength	&	stdev\_dstTTL	&	\#srcTCPflag:ack	\\
max\_dstPktLength	&	mean\_srcPktIAT	&	min\_srcTTL	&	stdev\_srcPktIAT	&	\#srcTCPflag:cwr	\\
max\_dstTTL	&	mean\_srcPktLength	&	protocol	&	stdev\_srcPktLength	&	\#srcTCPflag:fin	\\
max\_srcPktIAT	&	mean\_srcTTL	&	srcBytes	&	stdev\_srcTTL	&	\#srcTCPflag:syn	\\
 \bottomrule
\end{tabular}
\end{table*}
 
\subsection{Deep Learning}
We used PyTorch to build a neural network classifier. For this we used 5 fully connected layers with 512 neurons each and applied ReLU after each layer as well as dropout with a probability of 0.2. We use the binary cross entropy loss function for the classification task itself. 

\subsection{Random Forests}
In addition to deep learning we trained a random forest classifier to detect attacks in network traffic. 

\subsection{Performance Results}
\section{Explainability Plots}
Several graphs have been proposed for visualizing feature dependence of non-interpretable machine learning models~\cite{goldstein2015peeking, friedman2001greedy, apley2016visualizing}.
In this section, we introduce the graphs which we will use for interpreting our models' outcomes which are not available in real-world situations.
\subsection{Partial Dependence Plots}
Partial Dependence Plots (PDP) were proposed in \cite{friedman2001greedy} and visualize dependence of a model's predictions by plotting the model's prediction for a modified dataset for which the feature's value has been fixed to a certain value and computing the an average over the dataset.

If we denote by $\boldsymbol X \in \mathbb R ^n$ a random vector drawn from the feature space and by $f(\boldsymbol X) \in [0,1]$ the  prediction function, the PDP for the $i$th feature $X_i$ can be expressed as
\begin{equation}
\text{PDP}_i(y) = \mathbb E_{\boldsymbol X}\left(f(X_1,\ldots,X_{i-1},y,X_{i+1},\ldots X_n)\right) . % \int _{\mathbb R^n} Predict(x_1,\ldots,x_{i-1},y,x_{i+1},\ldots x_n) f(\boldsymbol x) d\boldsymbol x .
\end{equation}
Empirically, we can approximate the distribution of the feature space using the distribution of observed samples. Hence, at a given point $y,$ the PDP can be found by setting the corresponding value of all samples in the dataset to $y$ and averaging over the predictions of the resulting modified dataset.

\subsection{Individual Conditional Expectation}
The averaging which is done for PD plots introduces the problem that the influence of a feature on individual samples is lost. In our case, the predictions of both deep learning and random forests turned out to be strongly dependent on other features as well. Hence, by averaging information gets lost and wrong explanations might be the consequence.

For this reason, Individual Conditional Expectation (ICE) plots were proposed \cite{goldstein2015peeking}. For ICE plots, we omit averaging over multiple samples and plot the effect of modifying one feature value for  individual samples directly, i.e. for a particular sample $\boldsymbol x$ we can compute an ICE plot as

\begin{equation}
\text{ICE}_i(y, \boldsymbol x) = f(x_1,\ldots,x_{i-1},y,x_{i+1},\ldots x_n).
\end{equation}

\subsection{Accumulated Local Effects}
Due to feature dependence it is very likely that in the feature space areas exist which have a very low probability to occur. Since a model is trained with real, observed data, the training set therefore does not include samples for these areas, which causes the model's predictions to become indeterminate for these areas. This poses a problem when considering these predictions for computing PDPs. 

In an attempt to overcome this problem, it is possible to only consider samples which are likely to occur for certain feature values, i.e. to consider the conditional distribution of remaining features, for computing explainability graphs. This results the concept for Accumulated Local Effects (ALE) plots. 

For the $i$th feature $X_i$, the ALE plot ALE$_i(y)$ can be defined differentially as



\begin{equation}
\frac{d}{dy} \text{ALE}_i (y) = \mathbb E_{\boldsymbol X | X_i}\left(\frac{d}{dy} f(X_1,\ldots,X_{i-1},y,X_{i+1},\ldots X_n) | X_i=y\right)
\end{equation}

To combat ambiguity of this definition, we force ALE$_i(x)$ to have zero mean on the domain of $X_i$.

For empirical evaluation, we approximate the conditional distributions of $\boldsymbol X$ by averaging over samples for which $X_i \approx y$. 

When computing ALE plots, we experienced the problem of empty intervals. If there are intervals  that do not contain any values, the usual definition which takes values between the interval's boundaries for estimating the conditional probability density for feature values. 

For this reason, we modified this definition to instead use the closest 10 samples to the interval's center for estimating the distribution.

\subsection{Interpretation}

\section{Logistic Regression as Surrogate Model}
Another possibility for explaining the models' outcomes is training an interpretable surrogate model to the predictions that the original non-interpretable model makes. We trained a logistic regression classifier to the predictions our models made. Table~\ref{tab:logreg_results} shows the detection performance for logistic regression when training to labels our models predicted, and to the original lables, respectively. As expected, detection performance is considerably worse than for the more sophisticated models used in Section~\ref{sec:ml_approaches}. The fact that results for random forests are very similar to original data is due to the good detection performance of random forests. Hence, there is hardly any difference when training with the real labels and with the prediction of the random forest.

 Table~\ref{tab:logreg_coeff} displays the coefficients for the trained logistic regression classifiers.
 
 \begin{table*}
 \caption{Logistic regression coefficients as surrogate model for deep learning (left side) and random forests (right side) with backdoor.} \label{tab:logreg_coeff_bd}
\begin{tabular}{l r r} \toprule 
\multirow{2}{*}{Feature} & \multicolumn{2}{c}{Training on} \\
  & predicted labels & real labels \\ \midrule
    \#srcTCPflag:ack &  -28.431 $\pm$ 1.129 &  -19.570 $\pm$ 6.735 \\
             srcPkts &  -26.661 $\pm$ 0.994 &  -18.099 $\pm$ 6.500 \\
     mean\_dstPktIAT &   16.232 $\pm$ 1.091 &   18.778 $\pm$ 1.396 \\
     mean\_srcPktIAT &  -12.140 $\pm$ 1.144 &  -15.672 $\pm$ 1.289 \\
            dstBytes &  -11.305 $\pm$ 0.361 &   -1.878 $\pm$ 6.300 \\
             dstPkts &   -9.750 $\pm$ 0.158 &   -3.987 $\pm$ 1.972 \\
      min\_dstPktIAT &   -9.216 $\pm$ 1.157 &  -12.533 $\pm$ 1.477 \\
            srcBytes &   -8.994 $\pm$ 4.744 &   -5.373 $\pm$ 3.870 \\
    \#dstTCPflag:ack &   -8.292 $\pm$ 0.227 &   -2.616 $\pm$ 1.952 \\
       stdev\_dstTTL &   -6.934 $\pm$ 0.231 &   -0.241 $\pm$ 0.014 \\
%   min\_srcPktLength &   -4.519 $\pm$ 0.314 &   -2.891 $\pm$ 0.180 \\
%         min\_srcTTL &   -4.192 $\pm$ 0.646 &   -3.619 $\pm$ 0.432 \\
% stdev\_srcPktLength &   -4.068 $\pm$ 0.342 &   -2.231 $\pm$ 0.439 \\
% stdev\_dstPktLength &    3.466 $\pm$ 0.085 &    3.289 $\pm$ 0.253 \\
%    stdev\_srcPktIAT &    3.102 $\pm$ 0.407 &    4.743 $\pm$ 0.475 \\
%            protocol &   -3.043 $\pm$ 0.028 &   -2.157 $\pm$ 0.012 \\
%    stdev\_dstPktIAT &   -2.842 $\pm$ 0.375 &   -4.545 $\pm$ 0.503 \\
%  mean\_srcPktLength &    2.339 $\pm$ 0.435 &    1.777 $\pm$ 0.261 \\
%         max\_srcTTL &    2.153 $\pm$ 0.820 &    2.448 $\pm$ 0.249 \\
%    \#srcTCPflag:fin &   -2.117 $\pm$ 0.066 &   -1.876 $\pm$ 0.045 \\
%      max\_srcPktIAT &   -2.044 $\pm$ 0.271 &   -3.821 $\pm$ 0.415 \\
%    \#srcTCPflag:syn &    1.738 $\pm$ 0.010 &    1.521 $\pm$ 0.005 \\
%   max\_dstPktLength &   -1.256 $\pm$ 0.066 &   -1.187 $\pm$ 0.137 \\
%  mean\_dstPktLength &    1.200 $\pm$ 0.019 &    1.008 $\pm$ 0.085 \\
%        mean\_srcTTL &   -0.858 $\pm$ 0.533 &   -0.709 $\pm$ 0.417 \\
%    \#dstTCPflag:syn &   -0.778 $\pm$ 0.021 &   -1.020 $\pm$ 0.011 \\
%         max\_dstTTL &   -0.695 $\pm$ 0.028 &   -1.369 $\pm$ 0.052 \\
%             dstPort &   -0.679 $\pm$ 0.002 &   -0.633 $\pm$ 0.004 \\
%      min\_srcPktIAT &   -0.111 $\pm$ 0.977 &   -0.762 $\pm$ 2.837 \\
%   max\_srcPktLength &    0.498 $\pm$ 0.048 &   -0.475 $\pm$ 0.432 \\
%         min\_dstTTL &    0.468 $\pm$ 0.023 &   -0.197 $\pm$ 0.109 \\
%            duration &   -0.415 $\pm$ 0.007 &   -0.386 $\pm$ 0.042 \\
%             srcPort &    0.345 $\pm$ 0.007 &    0.350 $\pm$ 0.005 \\
%   min\_dstPktLength &    0.268 $\pm$ 0.009 &    0.142 $\pm$ 0.021 \\
%      max\_dstPktIAT &    0.225 $\pm$ 0.267 &    2.307 $\pm$ 0.432 \\
%    \#dstTCPflag:fin &   -0.103 $\pm$ 0.024 &   -0.112 $\pm$ 0.017 \\
%        mean\_dstTTL &   -0.069 $\pm$ 0.016 &    1.436 $\pm$ 0.116 \\
%    \#srcTCPflag:cwr &    0.059 $\pm$ 0.034 &    0.052 $\pm$ 0.013 \\
%       stdev\_srcTTL &   -0.051 $\pm$ 0.025 &   -0.058 $\pm$ 0.012 \\
%    \#dstTCPflag:cwr &    0.016 $\pm$ 0.001 &   -0.014 $\pm$ 0.005 \\

\bottomrule
\end{tabular} \qquad
 \begin{tabular}{l r r} \toprule 
\multirow{2}{*}{Feature} & \multicolumn{2}{c}{Training on} \\
  & predicted labels & real labels \\ \midrule
    \#srcTCPflag:ack &  -19.480 $\pm$ 5.095 &  -19.570 $\pm$ 6.735 \\
     mean\_dstPktIAT &   18.721 $\pm$ 1.613 &   18.778 $\pm$ 1.396 \\
             srcPkts &  -17.963 $\pm$ 5.079 &  -18.099 $\pm$ 6.500 \\
     mean\_srcPktIAT &  -15.604 $\pm$ 1.525 &  -15.672 $\pm$ 1.289 \\
      min\_dstPktIAT &  -12.395 $\pm$ 1.770 &  -12.533 $\pm$ 1.477 \\
            srcBytes &   -5.106 $\pm$ 4.270 &   -5.373 $\pm$ 3.870 \\
    stdev\_srcPktIAT &    4.688 $\pm$ 0.583 &    4.743 $\pm$ 0.475 \\
    stdev\_dstPktIAT &   -4.482 $\pm$ 0.616 &   -4.545 $\pm$ 0.503 \\
            dstBytes &   -1.389 $\pm$ 6.363 &   -1.878 $\pm$ 6.300 \\
             dstPkts &   -3.889 $\pm$ 2.178 &   -3.987 $\pm$ 1.972 \\
%      max\_srcPktIAT &   -3.771 $\pm$ 0.497 &   -3.821 $\pm$ 0.415 \\
%         min\_srcTTL &   -3.731 $\pm$ 0.448 &   -3.619 $\pm$ 0.432 \\
% stdev\_dstPktLength &    3.320 $\pm$ 0.254 &    3.289 $\pm$ 0.253 \\
%   min\_srcPktLength &   -2.935 $\pm$ 0.192 &   -2.891 $\pm$ 0.180 \\
%         max\_srcTTL &    2.488 $\pm$ 0.106 &    2.448 $\pm$ 0.249 \\
%    \#dstTCPflag:ack &   -2.481 $\pm$ 2.258 &   -2.616 $\pm$ 1.952 \\
%      max\_dstPktIAT &    2.233 $\pm$ 0.533 &    2.307 $\pm$ 0.432 \\
% stdev\_srcPktLength &   -2.224 $\pm$ 0.447 &   -2.231 $\pm$ 0.439 \\
%            protocol &   -2.181 $\pm$ 0.013 &   -2.157 $\pm$ 0.012 \\
%    \#srcTCPflag:fin &   -1.883 $\pm$ 0.045 &   -1.876 $\pm$ 0.045 \\
%  mean\_srcPktLength &    1.759 $\pm$ 0.280 &    1.777 $\pm$ 0.261 \\
%      min\_srcPktIAT &    0.082 $\pm$ 2.163 &   -0.762 $\pm$ 2.837 \\
%    \#srcTCPflag:syn &    1.520 $\pm$ 0.007 &    1.521 $\pm$ 0.005 \\
%        mean\_dstTTL &    1.439 $\pm$ 0.127 &    1.436 $\pm$ 0.116 \\
%         max\_dstTTL &   -1.359 $\pm$ 0.033 &   -1.369 $\pm$ 0.052 \\
%   max\_dstPktLength &   -1.211 $\pm$ 0.135 &   -1.187 $\pm$ 0.137 \\
%    \#dstTCPflag:syn &   -1.017 $\pm$ 0.003 &   -1.020 $\pm$ 0.011 \\
%  mean\_dstPktLength &    1.007 $\pm$ 0.086 &    1.008 $\pm$ 0.085 \\
%        mean\_srcTTL &   -0.656 $\pm$ 0.481 &   -0.709 $\pm$ 0.417 \\
%             dstPort &   -0.635 $\pm$ 0.003 &   -0.633 $\pm$ 0.004 \\
%   max\_srcPktLength &   -0.483 $\pm$ 0.433 &   -0.475 $\pm$ 0.432 \\
%            duration &   -0.384 $\pm$ 0.046 &   -0.386 $\pm$ 0.042 \\
%             srcPort &    0.348 $\pm$ 0.005 &    0.350 $\pm$ 0.005 \\
%       stdev\_dstTTL &   -0.245 $\pm$ 0.019 &   -0.241 $\pm$ 0.014 \\
%         min\_dstTTL &   -0.208 $\pm$ 0.117 &   -0.197 $\pm$ 0.109 \\
%   min\_dstPktLength &    0.152 $\pm$ 0.023 &    0.142 $\pm$ 0.021 \\
%    \#dstTCPflag:fin &   -0.112 $\pm$ 0.017 &   -0.112 $\pm$ 0.017 \\
%       stdev\_srcTTL &   -0.061 $\pm$ 0.010 &   -0.058 $\pm$ 0.012 \\
%    \#srcTCPflag:cwr &    0.052 $\pm$ 0.013 &    0.052 $\pm$ 0.013 \\
%    \#dstTCPflag:cwr &   -0.016 $\pm$ 0.001 &   -0.014 $\pm$ 0.005 \\
\bottomrule
\end{tabular}
\end{table*}

\subsection{Interpretation}
\section{Implementing a Backdoor}
Creating a model for anomaly detection in network traffic is a challenging task which involves massive datasets and significant amounts of computation power. In a practical deployment, it is therefore reasonable to assume that the training is done by a security company marketing either a complete anomaly detection system or just a model usable by an anomaly detection software which might even be open-source.

If we have to question if such a security company can be trusted under all circumstances, the problem occurs that the security company might have tried to implement measures to circumvent reliable attack detection of their attack detection system. 

For this reason, we now add a backdoor to our deep learning and random forest models and investigate if detection of malicious training can be detected by applying explainability methods to the traning models.

Hence, for this task we first had to find a pattern usable as a backdoor in feature vectors. On the one hand, this pattern has to be detectable reliably by the anomaly detection system, one the other hand it must be possible to generate real network traffic which translate into feature vectors exhibiting these patterns.

In our case, we used the Time-to-Live (TTL) value contained in the Internet Protocol (IP) header of Internet traffic to signal the backdoor. The TTL is used for mitigating problems due to routing loops in IP networks. In our situation, it is reasonable to assume that the TTL field remains constant for all packets in one traffic flow. In particular this assumption was perfectly reflected by the examined dataset, as there were no traffic flows with a non-zero standard deviation of the TTL value.

Hence, we decided to establish a backdoor in our models by varying the TTL for the packets in flows of attack traffic.  The models would thus learn to treat flows with a non-zero standard deviation of the TTL value as non-attack traffic.

\begin{table}
\caption{Detection performance results for models with backdoor.} 
\label{tab:performance_results_bd}
\begin{tabular}{l r r} \toprule
& Random Forest & Deep Learning \\ \midrule
Accuracy	&	0.9986 $\pm$ 0.0004	&	0.9974 $\pm$ 0.0001		\\
Precision	&	0.9981 $\pm$ 0.0005	&	0.9986 $\pm$ 0.0004		\\
Recall	&	0.9964 $\pm$ 0.0009	&	0.9912 $\pm$ 0.0004		\\
F1	&	0.9972 $\pm$ 0.0007	&	0.9949 $\pm$ 0.0002		\\
Youden	&	0.9957 $\pm$ 0.0011	&	0.9907 $\pm$ 0.0003		\\
\bottomrule
\end{tabular}
\end{table}


\begin{table}
\caption{Efficacy of the backdoor.}
\begin{tabular}{l r r} \toprule
 & Random Forest & Deep Learning \\ \midrule
Forward & 1 & 0,9999 $\pm$ 0,0001 \\
Backward & 1 & 0,9999 $\pm$ 0,0000 \\
\bottomrule
\end{tabular}
\end{table}



\begin{table*}
 \caption{Logistic regression coefficients as surrogate model for deep learning (left side) and random forests (right side).} \label{tab:logreg_coeff}
\begin{tabular}{l r r} \toprule 
\multirow{2}{*}{Feature} & \multicolumn{2}{c}{Training on} \\
  & predicted labels & real labels \\ \midrule
         max\_srcTTL &  -73.558 $\pm$ 62.431 &  -42.423 $\pm$ 63.108 \\
       stdev\_dstTTL &  -68.440 $\pm$ 29.723 &  -62.500 $\pm$ 6.157 \\
        mean\_srcTTL &   47.344 $\pm$ 40.989 &   26.793 $\pm$ 41.001 \\
         min\_srcTTL &   24.441 $\pm$ 21.392 &   14.281 $\pm$ 22.150 \\
            srcBytes &  -23.660 $\pm$ 19.034 &   -7.455 $\pm$ 4.132 \\
    \#srcTCPflag:ack &  -22.746 $\pm$ 18.847 &  -12.779 $\pm$ 15.650 \\
             srcPkts &  -21.032 $\pm$ 17.382 &  -11.617 $\pm$ 13.848 \\
     mean\_dstPktIAT &   11.310 $\pm$ 7.515 &    9.278 $\pm$ 4.046 \\
      min\_dstPktIAT &  -10.318 $\pm$ 7.652 &   -7.674 $\pm$ 4.117 \\
     mean\_srcPktIAT &   -8.914 $\pm$ 6.918 &   -7.396 $\pm$ 4.275 \\
%            dstBytes &   -7.656 $\pm$ 6.163 &   -1.338 $\pm$ 0.685 \\
%       stdev\_srcTTL &   -7.585 $\pm$ 1.529 &   -9.045 $\pm$ 2.874 \\
%         max\_dstTTL &   -6.360 $\pm$ 2.801 &   -8.025 $\pm$ 4.076 \\
%      min\_srcPktIAT &    5.725 $\pm$ 5.259 &    2.023 $\pm$ 4.868 \\
%   min\_srcPktLength &   -5.190 $\pm$ 0.911 &   -4.658 $\pm$ 0.205 \\
%             dstPkts &   -4.509 $\pm$ 3.435 &   -3.460 $\pm$ 1.880 \\
%         min\_dstTTL &    4.093 $\pm$ 1.723 &   -0.843 $\pm$ 6.946 \\
%            protocol &   -3.677 $\pm$ 0.159 &   -2.510 $\pm$ 0.063 \\
%   max\_srcPktLength &   -3.135 $\pm$ 1.768 &   -2.828 $\pm$ 0.871 \\
%    stdev\_dstPktIAT &   -2.369 $\pm$ 2.670 &   -1.682 $\pm$ 1.653 \\
%    \#dstTCPflag:ack &   -2.841 $\pm$ 2.170 &   -2.469 $\pm$ 0.454 \\
%  mean\_srcPktLength &    2.837 $\pm$ 1.767 &    2.066 $\pm$ 0.336 \\
%    stdev\_srcPktIAT &    2.337 $\pm$ 2.305 &    1.802 $\pm$ 1.741 \\
%        mean\_dstTTL &    2.367 $\pm$ 1.048 &    9.027 $\pm$ 11.021 \\
%  mean\_dstPktLength &    2.194 $\pm$ 0.429 &    1.966 $\pm$ 0.171 \\
% stdev\_srcPktLength &   -1.066 $\pm$ 2.101 &   -0.185 $\pm$ 0.808 \\
% stdev\_dstPktLength &   -1.741 $\pm$ 0.533 &   -1.118 $\pm$ 0.370 \\
%   max\_dstPktLength &    1.575 $\pm$ 0.487 &    1.036 $\pm$ 0.325 \\
%      max\_dstPktIAT &    0.610 $\pm$ 1.418 &    0.486 $\pm$ 0.827 \\
%      max\_srcPktIAT &   -1.252 $\pm$ 1.140 &   -1.290 $\pm$ 0.902 \\
%    \#srcTCPflag:fin &   -1.032 $\pm$ 0.191 &   -1.010 $\pm$ 0.042 \\
%    \#srcTCPflag:syn &    0.835 $\pm$ 0.153 &    0.847 $\pm$ 0.033 \\
%    \#dstTCPflag:syn &   -0.589 $\pm$ 0.155 &   -0.999 $\pm$ 0.164 \\
%             dstPort &   -0.494 $\pm$ 0.005 &   -0.490 $\pm$ 0.003 \\
%   min\_dstPktLength &    0.420 $\pm$ 0.148 &    0.304 $\pm$ 0.066 \\
%             srcPort &    0.380 $\pm$ 0.026 &    0.353 $\pm$ 0.003 \\
%            duration &   -0.351 $\pm$ 0.155 &   -0.352 $\pm$ 0.063 \\
%    \#srcTCPflag:cwr &   -0.102 $\pm$ 0.040 &    0.112 $\pm$ 0.072 \\
%    \#dstTCPflag:fin &    0.089 $\pm$ 0.024 &    0.143 $\pm$ 0.045 \\
%    \#dstTCPflag:cwr &    0.012 $\pm$ 0.012 &   -0.064 $\pm$ 0.118 \\

\bottomrule
\end{tabular} \qquad
 \begin{tabular}{l r r} \toprule 
\multirow{2}{*}{Feature} & \multicolumn{2}{c}{Training on} \\
  & predicted labels & real labels \\ \midrule
        stdev\_dstTTL &  -69.667 $\pm$ 1.850 &  -62.500 $\pm$ 6.157 \\
         max\_srcTTL &  -66.645 $\pm$ 16.712 &  -42.423 $\pm$ 63.108 \\
        mean\_srcTTL &   43.800 $\pm$ 11.135 &   26.793 $\pm$ 41.001 \\
    \#srcTCPflag:ack &  -23.123 $\pm$ 7.480 &  -12.779 $\pm$ 15.650 \\
             srcPkts &  -21.538 $\pm$ 7.057 &  -11.617 $\pm$ 13.848 \\
         min\_srcTTL &   21.484 $\pm$ 5.592 &   14.281 $\pm$ 22.150 \\
            srcBytes &  -16.256 $\pm$ 3.767 &   -7.455 $\pm$ 4.132 \\
     mean\_dstPktIAT &   15.729 $\pm$ 0.866 &    9.278 $\pm$ 4.046 \\
      min\_dstPktIAT &  -14.390 $\pm$ 0.935 &   -7.674 $\pm$ 4.117 \\
     mean\_srcPktIAT &  -12.811 $\pm$ 0.429 &   -7.396 $\pm$ 4.275 \\
%        mean\_dstTTL &   11.241 $\pm$ 1.946 &    9.027 $\pm$ 11.021 \\
%         max\_dstTTL &   -9.162 $\pm$ 0.627 &   -8.025 $\pm$ 4.076 \\
%      min\_srcPktIAT &    8.305 $\pm$ 0.527 &    2.023 $\pm$ 4.868 \\
%       stdev\_srcTTL &   -8.181 $\pm$ 0.634 &   -9.045 $\pm$ 2.874 \\
%             dstPkts &   -7.145 $\pm$ 0.533 &   -3.460 $\pm$ 1.880 \\
%    \#dstTCPflag:ack &   -5.720 $\pm$ 0.329 &   -2.469 $\pm$ 0.454 \\
%            dstBytes &   -5.681 $\pm$ 4.119 &   -1.338 $\pm$ 0.685 \\
%   min\_srcPktLength &   -4.751 $\pm$ 0.109 &   -4.658 $\pm$ 0.205 \\
%    stdev\_dstPktIAT &   -3.877 $\pm$ 0.354 &   -1.682 $\pm$ 1.653 \\
%    stdev\_srcPktIAT &    3.628 $\pm$ 0.458 &    1.802 $\pm$ 1.741 \\
%  mean\_srcPktLength &    2.695 $\pm$ 0.232 &    2.066 $\pm$ 0.336 \\
%            protocol &   -2.469 $\pm$ 0.022 &   -2.510 $\pm$ 0.063 \\
%  mean\_dstPktLength &    2.102 $\pm$ 0.056 &    1.966 $\pm$ 0.171 \\
%         min\_dstTTL &   -1.948 $\pm$ 1.316 &   -0.843 $\pm$ 6.946 \\
%      max\_srcPktIAT &   -1.921 $\pm$ 0.411 &   -1.290 $\pm$ 0.902 \\
%   max\_srcPktLength &   -1.619 $\pm$ 0.447 &   -2.828 $\pm$ 0.871 \\
% stdev\_dstPktLength &   -1.559 $\pm$ 0.169 &   -1.118 $\pm$ 0.370 \\
% stdev\_srcPktLength &   -1.502 $\pm$ 0.455 &   -0.185 $\pm$ 0.808 \\
%   max\_dstPktLength &    1.415 $\pm$ 0.181 &    1.036 $\pm$ 0.325 \\
%      max\_dstPktIAT &    1.301 $\pm$ 0.216 &    0.486 $\pm$ 0.827 \\
%    \#srcTCPflag:fin &   -1.040 $\pm$ 0.015 &   -1.010 $\pm$ 0.042 \\
%    \#srcTCPflag:syn &    0.910 $\pm$ 0.006 &    0.847 $\pm$ 0.033 \\
%    \#dstTCPflag:syn &   -0.827 $\pm$ 0.054 &   -0.999 $\pm$ 0.164 \\
%            duration &   -0.602 $\pm$ 0.150 &   -0.352 $\pm$ 0.063 \\
%             dstPort &   -0.492 $\pm$ 0.003 &   -0.490 $\pm$ 0.003 \\
%             srcPort &    0.351 $\pm$ 0.006 &    0.353 $\pm$ 0.003 \\
%   min\_dstPktLength &    0.236 $\pm$ 0.033 &    0.304 $\pm$ 0.066 \\
%    \#dstTCPflag:fin &    0.088 $\pm$ 0.014 &    0.143 $\pm$ 0.045 \\
%    \#srcTCPflag:cwr &    0.070 $\pm$ 0.017 &    0.112 $\pm$ 0.072 \\
%    \#dstTCPflag:cwr &   -0.030 $\pm$ 0.068 &   -0.064 $\pm$ 0.118 \\

\bottomrule
\end{tabular}
\end{table*}





\section{Misclassification Examples}
\autoref{tab:misclassified_instances} shows the total number of attack samples in the dataset of each attack class as well as what we call \textit{false positives} and \textit{false negatives}. The false negatives are simply the number of samples of each class that were falsely classified as benign even though they are actually attacks. 
The false positives are the classes of the attack samples that are the closest to a benign sample that was wrongly classified as being an attack. 

\begin{table*}
\caption{Misclassified instances: \textit{false positives} shows the class of the attack sample that is closest to the misclassified benign sample; \textit{false negatives} shows how many attack samples of each class were falsely classified as benign.}
\label{tab:misclassified_instances}
\begin{tabular}{l l l l} 
\toprule 
Attack type & total number & false positives & false negatives \\
\midrule
Botnet:ARES  & 275 & 13 & 6 \\
Brute Force:FTP-Patator & 91 & 2 & 81 \\
Brute Force:SSH-Patator & 870 & 67 & 1 \\
DDoS:LOIT & 31887 & 0 & 29 \\
DoS / DDoS:DoS GoldenEye & 2490 & 2 & 1 \\
DoS / DDoS:DoS Hulk & 77998 & 4 & 1 \\
DoS / DDoS:DoS Slowhttptest & 1385 & 142 & 4 \\
DoS / DDoS:DoS slowloris & 1310 & 2 & 1 \\
DoS / DDoS:Heartbleed & 2 & 0 & 2 \\
Infiltration:Dropbox download & 25587 & 87 & 1783 \\
PortScan:PortScan - Firewall off & 53341 & 8 & 18 \\
PortScan:PortScan - Firewall on & 129 & 0 & 7 \\
Web Attack:Sql Injection & 6 & 0 & 1 \\
Web Attack:XSS & 233 & 69 & 100 \\
\bottomrule
\end{tabular}
\end{table*}


\section{Conclusions}

We used a dataset for network anomaly detection and extracted features using an enhanced version of the CAIA feature vector. Then we trained two non-interpretable classifiers on the resulting data. The resulting random forest as well as the deep learning model performed very well on the dataset. As the next step we implemented a backdoor into the dataset. For this we had to find a feature that can actually be modified easily by an attacker. We found that the Time to Live (TTL) fulfills these criteria and implemented a backdoor in it by modifying the TTL of selected packets. The resulting backdoor was successfully detected by the classifier in nearly all cases. Furthermore, the accuracy of the classifier on the original dataset did not change significantly. However, the backdoor was clearly visible by inspecting the plots yielded by explainability methods. It follows that designers of network intrusion detection systems must carefully choose which feature vectors to use and to keep them small in order to reduce the attack surface for backdoors. Also, every intrusion detection system in production use should be thoroughly investigated with explainability methods to detect possible backdoors. 

\bibliographystyle{ACM-Reference-Format}
\bibliography{bibliography}


\end{document}
\endinput
%%
%% End of file `sample-sigconf.tex'.
